The DL project
Finding inconsistencies and redundancies in an ontology.
Install Protege from http://protege.stanford.edu/ (WebProtege is too limited for our purposes). Activate a reasoner plugin as part of Protege (FACT++ or Pellet will do), download the pizza.owl ontology from here and open it in Protege. A useful way to navigate around your ontology is the "export -> OWLDoc" option.
TASK 0: inspect the class hierarchy windows -> views -> class views -> class hierarchy compare asserted vs. inferred (before and after you've run a reasoner)
TASK 1: find which classes are inconsistent, find ways to resolve the inconsistency, implement them in the Protege editor, and save the result as your own consistent-pizza.owl file.
TASK 2: find if there are places where the pizza.owl ontology contains redundant statements. If you find any, report them and explain why they are redundant. Notice in the pane for object-property-hierarchy that object-properties such as isIngredientOf can have characteristics (such as being transitive) (you can also use window -> views -> object property ). Such characteristics are often the source of redundant statements. But of course there are other sources, such as redundant number restrictions, redundant subclass statements, redundant class-intersections, and many others.
TASK 3: Extend the ontology with some obvious redundant statements, and save the result as redundant-pizza.owl. You may want to derive all inferred subsumption statements to check that your added statements were indeed only redundant: use file -> export inferred statements..., and choose a syntax you can read, such as Manchester syntax, or OWL functional syntax, or even the LaTeX syntax (and format it).
Submit the following files
a text-file (one page will do) explaining
which redundant statements you found in pizza.owl, or saying that you think pizza.owl is not redundant,
An explanation and justification of the changes you made to pizza.owl to create consistent-pizza.owl
An explanation and justification of the changes you made to pizza.owl to create redundant-pizza.owl
consistent-pizza.owl
redundant-pizza.owl
Submit these as a single .zip file here. Make sure that your names are clearly indicated in the name of the zip-file that you upload. The deadline is 10 Oct, 23h59.
Resources
A Protege tutorial using the Pizza ontology is at http://protegewiki.stanford.edu/wiki/Protege4Pizzas10Minutes. The tutorial is about building the ontology, while you already get the constructed ontology and are asked to inspect and modify it, but it will help you navigate around Protege.
Usage of Protege is pretty self-explanatory (once you understand the basics of Description Logics), but there is also plenty of instruction manuals around on the Web, for example http://bmir-stage.stanford.edu/doc/owl/getting-started.html
