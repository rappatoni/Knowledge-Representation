The Qualitative Reasoning practical
The description of what to do for the Qualitative Reasoning practical, and what to hand in.
1. Goals include
Get familiar with basic concepts relevant to Qualitative Reasoning.
Work with various notions of causality, especially with Direct (Influence, I+/I–) and Indirect (Proportionality, P+/P–) influences, and feedback loops.
Represent and automate reasoning about dynamic systems using qualitative representations.
2. System description
Consider containers such as bathtubs and kitchen sinks, which can be filled with water (e.g. via a tab) and emptied (e.g. via a drain). What behaviours can occur with such container systems? Your assignment is to create a programme that reasons about the possible behaviours of such systems and generates a state-graph showing all possible behaviours.
Start with the following details:
Quantities
Inflow (of water into the container)
Outflow (of water out of the container)
Volume (of the water in the container)
Quantity spaces
Inflow: [0, +]
Outflow and Volume: [0, +, max]
Dependencies
I+(Inflow, Volume) - The amount of inflow increases the volume
I-(Outflow, Volume) - The amount of outflow decreases the volume
P+(Volume, Outflow) - Outflow changes are proportional to volume changes
VC(Volume(max), Outflow(max)) - The outflow is at its highest value (max), when the volume is at its highest value (also max).
VC(Volume(0), Outflow(0)) - There is no outflow, when there is no volume.
3. Assumptions
When developing the solution to this problem, you may discover that varying solutions can be postulated, depending on assumptions you make about the system. One assumption concerns the inflow which is exogenously defined. How will it behave? You may choose assumptions at your discretion. Describe your choices and their impact in the report you submit.
4. Assignment main steps
Create drawings of the causal model active for this system and the expected state-graph. Show all the states with their unique value set and the state-transitions. You can use paper & pencil or a computer tool of your preference.
Develop and implement (using your favourite programming language) an algorithm that uses the details discussed above, and creates a state-graph showing all the behavioural states of the container system (and as envisioned in A). Important here is your representation of states and state-transitions, and how you deploy the calculi for the different dependency types.
Augment the algorithm such that it generates an insightful trace of results inferred by the algorithm. It will be helpful to distinguish between intra-state (within a state) and inter-state (between states) conclusions.
Describe and submit your results. Show and explain how your algorithm works. Include the outputs of the algorithm, notably the trace and the state-graph.
You can choose to make the additional part of the assignment (see 5. Extra below).
You have to upload your results here by Friday, Oct 20, 23h59 at the latest. Make sure that your names are clearly indicated in the name of the zip-file that you upload.
5. Extra
The details explaining the working of the container can be represented more accurately by including column height and bottom pressure (of the column). Augment your approach by including these intermediate quantities. Start with the following details:
Quantities
Height (of the water column in the container)
Pressure (of the water column at the bottom of the container)
Quantity spaces
Height and Pressure: [0, +, max]
Dependencies
P+(Volume, Height) - Height changes are proportional to volume changes
P+(Height, Pressure) - Pressure changes are proportional to height changes
Additional issues
Instead of volume, it is the pressure that determines the outflow
Particular values, such as 0 and max correspond for volume, height, pressure and outflow.
6. Scoring QR assignment (total 100 points)
Self created causal model and state-graph (A) (15 points)
Algorithm working and well described (B & D) (20 points)
Algorithm output state-graph and well described (20 points)
Assumptions well described (10 points)
Insightful trace (C) (20 points)
Extra details (E) (15 points)

Python network graph draw: https://plot.ly/python/network-graphs/
